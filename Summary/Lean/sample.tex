\documentclass[letterpaper,13pt]{texMemo}
\usepackage[american]{babel}
\usepackage{natbib}
\usepackage{setspace}

\memoto{Roby Conner, Writing 327 Instructor}
\memofrom{Jacques Uber, Oregon State University Student}
\memodate{Feb 12, 2012}
\memosubject{Summary of State-of-the-art in lean design engineering}

\begin{document}
\singlespace
\begin{center}
\large {Memorandum}
\end{center}
\setlength{\topmargin}{0in}
\maketitle
\subsubsection*{Primary Research Question}
\noindent
In the Literature review \emph{State-of-the-art in lean design engineering: a literature review on
white collar lean} the authors, T Baines, H Lightfoot, G M Williams, and R Greenough, sought to answer the
question: What is the meaning of the term "Lean" in the context of product development (PD) and how
can Lean be used to improve PD? The review was targeted
towards PD in the aerospace industry. There were few articles about applying Lean to PD, but
existing evidence determined that applying Lean to product design can be economically beneficial.

\subsubsection*{Findings}
The authors had six main findings:
\begin{enumerate}
\item
The term "Lean" was developed by the manufacturing industry. Because the process of design
is different than manufacturing, the definition of "Lean" is starting to change. This is causing
misunderstandings between researchers and practitioners when discussing Lean.

\item
Lean is a viable option for PD, specifically in the aerospace industry, but to what
extent is still unknown.

\item
The Lean used in manufacturing is well defined. The Lean that could be used in product
design is not yet well defined. This vagueness can be attributed to Lean's origin, manufacturing, and
only recent application to PD.

\item
To help guide application, the review suggest that the methods used by Toyota (credited for
developing modern Lean techniques) could be applied to PD. One method would make individual
functions of the PD process ensure certain parts of the final product.

\item
When Lean is used in PD the leadership provided by the chief engineer (CE) is critical.

\item
Organizations may need to completely re-organize to successfully implement Lean in their development
process.
\end{enumerate}

\subsubsection*{Issues}
\noindent
The review found that information about how an organization's management should implement Lean in
PD is not readily accessible. They also found when Lean is used in PD, value creation is not well defined.

\end{document}
