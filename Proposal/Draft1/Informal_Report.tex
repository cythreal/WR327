\documentclass[letterpaper,11pt]{texMemo}

\usepackage{ifpdf}
\usepackage{mla}
\usepackage{setspace}
\bibliography{bibtex.bi}

%\memotitle{Memorandum}
\memoto{Roby Conner, Writing 327 Instructor}
\memofrom{Jacques Uber, Oregon State University Student}
\memodate{Feb 12, 2012}
\memosubject{Proposal to investigate imporovements to Tor}
%\memologo{Memorandum}

\begin{document}
\singlespace
\maketitle

\section*{Introduction}

I will investigate how engineers are working to improve Tor. Specifically, I will focus on the
improvements to latency and congestion, and the methods used to improve user anonymity.

The literature review will include all scholarly articles on Tor found via the Academic Search
Premier database using the keywords Tor, Improvements, Congestion, Fair, Timing Attacks,
Anonymity and published between the years 2009 and 2012.

There will be three section in the document. The first section will introduce core concepts used
to implement Tor. The second will investigate techniques used to compromise user
anonymity. The third will investigate network latency and congestion.

\section*{Background}

    o The bulk of your research will go here to explain the context, significance,
    and source of the problem you intend to research. You must cite all
    information taken from your research.
    \subsection*{What is Tor?}
    Tor is an overlay network that enables users to access services without revealing their IP
    address (Internet Protocol Address). Tor was originally developed by the Navy and is used by
    militaries, journalist, law enforcement, activists, and the average internet user; This diversity
    in users helps users of Tor remain anonymous while using the internet. (Tor website) % How to
    cite website?

    Tor is made up of two types of nodes: Onion Routers (OR) and Onion Proxies (OP). When a user wants to use
    Tor to connect to a service on the internet it first downloads a directory containing the IP
    addresses of all Onion Routers. The user then sets up an Onion Proxy. This OP is the users entry
    point into the Tor network. The user then constructs a circuit using a number of (usually three)
    randomly chosen OR's. To construct a circuit, the user contacts the first OR using the OP as a
    middle man and set's up an ecrypted connection to that OR. Then, using the first OR as a proxy,
    the user builds a second encrypted channel with the second OR. It builds a similar connection
    with the third OR using the first and second OR's as forwarders. The final OR speaks to the
    service the user is trying to contact and relays any response back through the cuircut the user
    created. Tor is known as the Onion Router because traffic destined to the final router is
    encrypted in layers.

    o Discuss history of the problem/topic
    Tor 

    o Demonstrate the need for the research
    Research Plan

    o The literature review will address the following research questions. Restate
    them to reflect your specific topic.

         What trends in research methods or results can be identified in the
        literature?

         Are there identifiable controversies within the literature?

         Are there identifiable weaknesses in the methods or conclusions in the
        literature?

         What areas/topics for future research are recommended in the
        literature?

    o Minimum of six (6) sources (peer-reviewed articles from academic, trade, or
    scientific journals)

         Include a brief description of each article (Title, author, topic)

         1-2 sentences should describe why the article is included

\section*{Research Plan}
Conclusion
    o Summary of Key Points

    o Request for Action/Approval
    Works Cited Page (MLA formatted)

\nocite{*}
\bibliographystyle{plain}   % (uses file "plain.bst")
\bibliography{refs}       % expects file "myrefs.bib"
\end{document}
