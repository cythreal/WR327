\documentclass[letterpaper,12pt]{texMemo}
\usepackage[american]{babel}
\usepackage{natbib}
\usepackage{setspace}

\memoto{Roby Conner, Writing 327 InstrucTor}
\memofrom{Jacques Uber, Oregon State University Student}
\memodate{Feb 12, 2012}
\memosubject{Proposal to investigate imporovements to Tor}

\begin{document}
\singlespace
\begin{center}
\large {\bf Memorandum}
\end{center}
\setlength{\topmargin}{0in}
\maketitle
%\section*{Introduction}

It is my goal to investigate how engineers are working to improve latency and congestion issues in Tor.

The literature review will include all scholarly articles on Tor found via the Academic Search
Premier database using the keywords Tor, Improvements, Congestion, Fair, Timing Attacks,
Anonymity and published between the years 2009 and 2012.

\subsubsection*{Organization}
There will be three section in the document. The first section will introduce core concepts used to
implement Tor. The second will investigate why latency and congestion exists and why it is a
problem. The third will investigate the proposals to improve latency and congestion.

\section*{Background}

    \subsection*{What is Tor?}
    Tor is an overlay network that enables users to use the Internet anonymously. Tor was originally
    developed by the Navy and is used by militaries, journalist, law enforcement, activists, and the
    average internet user \citep[2]{Tor:web}. To ensure a user's privacy and anonymity Tor uses multiple
    layers of encryption while routing cells of data and is sometimes referred to as the "Onion
    Router". By building circuits between multiple nodes, a user routes her traffic through the
    network. Tor is an overlay network. This means that packets are routed and scheduled. While the
    majority of networks are packet switched, Tor is built on a circuit switching scheme.

    \subsection*{Why is Tor important?}
    Tor has many legitimate uses. ONI, the OpenNet Initiative, reported that "the Middle East and
    North Africa is one of the most heavily censored regions in the world". It also claimed that it
    "conducted tests for technical Internet filtering in all of the countries in the Middle East and
    North Africa between 2008 and 2009. Test results prove that the governments and Internet service
    providers (ISPs) censor content deemed politically sensitive; critical of governments, leaders
    or ruling families; morally offensive; or in violation of public ethics and order."
    (\citeauthor[6]{ONI}) Tor can bypass these filters ensuring a freedom of speech and freedom of
    expression on the Internet.

    \subsection*{Congestion and Delay}
    As of 2010 users on the Tor network have experienced network delay. \citeauthor[]{delay} ask the
    questions: why is there delay in the network, and where is the delay taking place? The Tor
    routers, as opposed to the proxies or the target service, seemed to be the cause. "Router delays
    are the principal contributors to delays in Tor. Some routers frequently introduce delays as
    high as a few seconds" (\citeauthor[3]{delay}). They used log files from network nodes that they
    controlled to measure "Total Delay" while making sure that delay caused by the target service was not
    included in the timing data.


\section*{Research Plan}

    Different protocols can cause congestion more than others. This is the focus of
    \citeauthor{analysis}. There is concern that bulk transfer protocols, like FTP (File Transfer
    Protocol) and P2P (Peer to Peer) protocols, are causing latency sensitive protocols, like ssh
    and HTTP, to become delayed and in some cases hard to use (\citeauthor[2]{analysis}). This
    problem is not new. Major ISPs (Internet Service Providers) have allowed their customers to have
    the ability to stream music and browse the web while also accommodating other services like FTP
    and BitTorent. This coexistence is normally achieved by packet shapers. A packet shapers looks
    at traffic (usually the source and destination port of the traffic) and gives bandwidth priority
    to latency sensitive protocols. This is not possible on the Tor network. The encryption that
    gives anonymity also stops the use of QoS (Quality of Service) mechanisms.

    Reworking how Tor schedules traffic is a possible solution to Congestion. When deciding when to
    forward a cell, a Tor Onion Router treats all data equally. Also, a Router will forward data for
    multiple circuits and it uses a Round Robin algorithm to determine which circuit it will
    service. This means that a circuit with data that tends to come in bursts will have the same
    priority as a circuit that contains a relatively continuous flow of data through it. This is not
    optimal because data that comes in bursts is usually sensitive to latency and should take
    priority over traffic that appears continuous (\citeauthor[2]{unfair}). There have been
    multiple scheduling schemes proposed to replace the Round Robin scheduler. A large part my literature
    review will be spent reviewing these scheduling algorithms and their effect on delay and latency.

\subsubsection*{White Papers}
\begin{itemize}
    \item
    \citeauthor*{unfair}: This paper explores the mechanisms for handling congestion
    and fairness and proposes a new scheduling algorithm.
    \item
    \citeauthor*{Tang}: This paper proposes a new scheduling algorithm.
    \item
    \citeauthor*{analysis}: This paper does analysis on what the Tor network is being used for and where congestion is taking place.
    \item
    \citeauthor*{delay}: This paper investigates where and how Tor is unfair to certain kinds of traffic.
    \item
    \citeauthor*{Bauer}: This paper discusses the consequences of replacing the Round Robin algorithm and
    proposes a new algorithm.
    \item
    \citeauthor*{Edman}: This paper explores modifications to Tor's path selection algorithm to help
    clients avoid observers from discovering their identities.
    \item
    \citeauthor*{Moore}: This paper describes a different idea of how to improve latency. The
    authors propose that traffic should be throttled at the ingress of the network.

\end{itemize}


\section*{Conclusion}
\begin{itemize}
    \item
    Tor is a valuable cyber anonymity tool for many disciplines.
    \item
    Tor has congestion and latency issues.
    \item
    There are many different proposed solutions to Tor's congestion and latency issues.
\end{itemize}
I am asking for approval to do a literature review on the ways engineers are addressing the
congestion and latency issues within the Tor network.

\bibliographystyle{mla-good}
\bibliography{sample}
\end{document} 
