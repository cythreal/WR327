\documentclass[letterpaper,13pt]{texMemo}
\usepackage[american]{babel}
\usepackage{natbib}
\usepackage{setspace}

\memoto{Roby Conner, Writing 327 Instructor}
\memofrom{Jacques Uber, Oregon State University Student}
\memodate{Feb 12, 2012}
\memosubject{Proposal to investigate imporovements to Tor}

\begin{document}
\begin{center}
\large {\bf Memorandum}
\end{center}
\singlespace
\setlength{\topmargin}{0in}
\maketitle
\section*{Introduction}
\noindent
It is my goal to investigate how engineers are working to improve latency and congestion issues in
Tor. The literature review will include all scholarly articles on Tor found via the Academic Search
Premier, IEEE, and ACM databases using the keywords Tor, Improvements, Congestion, Fair, Timing
Attacks, Anonymity and published between the years 2009 and 2012.

\noindent
\\There will be three section in this document. The first section will introduce core concepts used to
implement Tor. The second will investigate why latency and congestion exists and why it is a
problem. The third will investigate the proposals to improve latency and congestion.

\section*{Background}

    \subsection*{What is Tor?}
    Tor enables users to use the Internet anonymously. Tor was originally developed by the Navy
    and is used by militaries, journalist, law enforcement, activists, and the average internet user
    \citep[2]{Tor:web}.
    %To ensure a user's privacy and anonymity Tor uses multiple layers of
    %encryption while routing cells of data. By building circuits between multiple nodes, a user
    %routes her traffic through the network. Tor is an overlay network. This means that packets are
    %routed and scheduled. While the Internet is packet switched, Tor is built on a circuit switching
    %scheme.
    "Tor is a volunteer-operated network of approximately 2,500 application-layer routers (also
    called relays or ORs). The network provides anonymity by forwarding traffic from clients (also
    called proxies or OPs) along a bidirectional anonymous circuit consisting of Tor routers. To
    conceal the identities of the communicants, Tor encrypts messages such that each relay can
    discern only the identities of the previous and next hops along the anonymous circuit. By
    default, Tor uses three-relay hops, consisting of a guard relay, a middle relay, and an exit
    relay." (\citeauthor[1]{Moore})

    \subsection*{Why is Tor important?}
    ONI, the OpenNet Initiative, reported that "the Middle East and North Africa is one of the most
    heavily censored regions in the world". ONI says that it "conducted tests for technical
    Internet filtering in all of the countries in the Middle East and North Africa between 2008 and
    2009. Test results prove that the governments and Internet service providers (ISPs) censor
    content deemed politically sensitive; critical of governments, leaders or ruling families;
    morally offensive; or in violation of public ethics and order" (\citeauthor[6]{ONI}). Tor can
    bypass these filters ensuring freedom of speech and expression on the Internet.

\section*{Research Plan}


    \subsection*{Congestion and Delay}
    As of 2010, users on the Tor network have experienced network delay. "Why are there delays in
    the network?" and "Where are the delays taking place?" are the questions asked by
    \citeauthor[]{delay}. "Router delays are the principal contributors to delays in Tor. Some
    routers frequently introduce delays as high as a few seconds" (\citeauthor[3]{delay}). They used
    log files from network nodes that they controlled to measure "Total Delay" while making sure
    that delay caused by the target service was not included in the timing data.

    \subsection*{Causes}
    Certain protocols can cause congestion more than others. This is the focus of
    \citeauthor{analysis}. There is concern that bulk transfer protocols, like FTP (File Transfer
    Protocol) and P2P (Peer to Peer) protocols, are causing latency sensitive protocols, like ssh
    and HTTP, to become delayed and in some cases hard to use (\citeauthor[2]{analysis}). This
    problem is not new. Major ISPs have allowed their customers to have the ability to stream music
    and browse the web while also accommodating other services like FTP and BitTorent. This
    coexistence is normally achieved by packet shapers. A packet shaper looks at the TCP source and
    destination port of traffic to determine what type of traffic it is. It can then give
    bandwidth priority to latency sensitive protocols. This is not possible on the inner nodes of
    the Tor network due to the encrypted nature of Tor.

    \subsection*{Ingress and Postgress Filtering}
    If a user is not using a natively encrypted service like HTTP or standard Bittorrent, it is
    possible to preform DPI (Deep Packet Inspection) on that traffic when it leaves or exits the
    network. It has been proposed that "exit relays could examine outgoing traffic and discard any
    detected BitTorrent packets." Blocking Bittorent outright might seem ironic and rate limiting
    seems to be a better option (\citeauthor[2]{Moore}).

    \subsection*{Scheduler}
    A possible solution to congestion is to rework how Tor schedules traffic. A Tor onion router
    treats all data equally when deciding when to forward a cell. Furthermore, a router will
    forward data for multiple circuits using a Round Robin algorithm to determine which circuit it
    will serve. This means that a circuit with data that comes in bursts will have the same priority
    as a circuit that contains a periodic flow of data. This is not optimal because data that comes
    in intermittently is usually sensitive to latency and should take priority over traffic that
    arrives at regular intervals (\citeauthor[2]{unfair}). There have been multiple scheduling
    schemes proposed to replace the Round Robin scheduler. A large part my literature review will be
    spent reviewing these scheduling algorithms and their effect on delay and latency.

\subsubsection*{White Papers}
\begin{itemize}
    \item
    \citeauthor*{unfair}: This paper explores the mechanisms for handling congestion
    and fairness and proposes a new scheduling algorithm.
    \item
    \citeauthor*{Tang}: This paper proposes a new scheduling algorithm.
    \item
    \citeauthor*{analysis}: This paper does analysis on what the Tor network is being used for and where congestion is taking place.
    \item
    \citeauthor*{delay}: This paper investigates where and how Tor is unfair to certain kinds of traffic.
    \item
    \citeauthor*{Bauer}: This paper discusses the consequences of replacing the Round Robin algorithm and
    proposes a new algorithm.
    \item
    \citeauthor*{Edman}: This paper explores modification to Tor's path selection algorithm to help
    clients prevent observers from discovering their identities.
    \item
    \citeauthor*{Moore}: This paper proposes that traffic should be throttled at the ingress of the
    network. The authors admit this is an unorthodox approach.

\end{itemize}


{\section*{Conclusion}}
\begin{itemize}
    \item
    Tor is a valuable cyber anonymity tool for many disciplines.
    \item
    Tor has congestion and latency issues.
    \item
    There are many different proposed solutions to Tor's congestion and latency issues.
\end{itemize}
I am asking for approval to do a literature review on the ways engineers are addressing the
congestion and latency issues within the Tor network.

\newpage
\bibliographystyle{mla-good}
\center \renewcommand\refname{Work Cited\\ }
\bibliography{sample}
\end{document}
