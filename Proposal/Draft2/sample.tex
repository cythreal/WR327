\documentclass[letterpaper,12pt]{texMemo}
\usepackage[american]{babel}
\usepackage{natbib}
\usepackage{setspace}

\memoto{Roby Conner, Writing 327 Instructor}
\memofrom{Jacques Uber, Oregon State University Student}
\memodate{Feb 12, 2012}
\memosubject{Proposal to investigate imporovements to Tor}

\begin{document}
\singlespace
\begin{center}
\large {\bf Memorandum}
\end{center}
\setlength{\topmargin}{0in}
\maketitle
%\section*{Introduction}

It is my goal to investigate how engineers are working to improve the latency and congestion issues in Tor.

The literature review will include all scholarly articles on Tor found via the Academic Search
Premier database using the keywords Tor, Improvements, Congestion, Fair, Timing Attacks,
Anonymity and published between the years 2009 and 2012.

There will be three section in the document. The first section will introduce core concepts used to
implement Tor. The second will investigate why latency and congestion exists. The third will
investigate the proposals to improve latency and congestion.

\section*{Background}

    \subsection*{What is Tor?}
    Tor is an overlay network that enables users to use the Internet anonymously. Tor was originally
    developed by the Navy and is used by militaries, journalist, law enforcement, activists, and the
    average internet user\citep[2]{tor:web}. To ensure a users privacy and anonymity Tor uses multiple
    layers of encryption while routing cells of data and is sometimes refered to as the "Onion Router".

    \subsection*{Why is Tor important?}
    Tor has many legititamte uses. ONI, the OpenNet Initiative, reported that "the Middle East and
    North Africa is one of the most heavily censored regions in the world". It also claimed that it
    "conducted tests for technical Internet filtering in all of the countries in the Middle East
    and North Africa between 2008 and 2009. Test results prove that the governments and Internet
    service providers (ISPs) censor content deemed politically sensitive; critical of governments,
    leaders or ruling families; morally offensive; or in violation of public ethics and order." Tor
    can bypass these filters ensuring a freedom of speech and freedom of expression on the Internet.


\section*{Research Plan}
    As of 2010 users on the Tor network have experienced network delay. Reason for network delay was the
    focus of \citeauthor[]{delay}.  Why there delay in the network, and
    where is the delay taking place? The routers themselves could be the cause. All traffic that
    goes through the network needs to pass through routers. It could be that there are bottle
    necks forming at certain nodes \citeauthor{delay}. The method used to test for what is causing the delays was
    to set up multiple Onion Routers and measure where bottle necks occur and record how the circuit
    selection algorithm is making node selections.

    Different protocols can cause congestion more than others. This is the focus of
    \citeauthor{analysis}.  There is growing concern that bulk transfer protocols, like Bittorent
    and other P2P (Peer to Peer) protocols, are causing latency sensitive protocols, like ssh and
    HTTP, to become delayed and in some cases hard to use (\citeauthor[2]{analysis}). This problem
    is not new.  Major ISPs (Internet Service Providers) have allowed 
    their customers have the ability to stream music and browse the web while also accomidating other
    services like FTP (File Transfer Protocol) and Bittorent. This coexistance is normally achieved
    by packet shapers. A packet shapers looks at traffic (usually the source and destination port of
    the traffic) and gives bandwidth priority to latency sensitive protocols. This is not possible on
    the Tor network. The encryption that gives anonymity also stops the use of QoS (Quality of
    Service) mechanisms.

    Reworking how Tor scheduals traffic is a possible solution to Congestion.  A Tor Onion Router
    treats all data equally. Also, a Router will forward data for multiple circuits and it uses a Round Robin
    algorithm to determine which circuit it will service. This means that circuits with data that
    tends to come in bursts will have the same priority as a circuit that has a relativly continuous
    flow of data through it. This is not optimal because data that comes in bursts is usually
    sensative to latency and should take priority over traffic that appears continuous.
    (\citeauthor[2]{unfair}).  There have been multiple schedualling schemes proposed to replace the
    Round Robin schedauler.

\begin{itemize}
    \item
    \citeauthor{unfair}: This paper explores the mechanisms for handling congestion
    and fairness and proposes a new scheduling algorithm.
    \item
    \citeauthor{Tang}: This paper proposes a new scheduling algorithm.
    \item
    \citeauthor{analysis}: This paper does analysis on what the tor network is being used for and where congestion is taking place.
    \item
    \citeauthor{delay}: This paper investigates where and how Tor is unfair to certain kinds of traffic.
    \item
    \citeauthor{Bauer}: This paper discusses the consiquences of replacing the Round Robin algorithm and
    proposes a new algorithm.
    \item
    \citeauthor{Edman}: This paper explores modifications to Tor's path selection algorithm to help
    clients avoid an observer from discovering their identities.
\end{itemize}


\section*{Conclusion}
\begin{itemize}
    \item
    Tor is a valuable cyber anonymity tool for many disciplines.
    \item
    Tor has congestion and latency issues.
    \item
    There are many different proposed solutions to Tor's congestion and latency issues.
\end{itemize}
I am asking for approval to do a literature review on the ways engineers are addressing the
congestion and latency issues within the Tor network.

\bibliographystyle{mla-good}
\bibliography{sample}
\end{document} 
