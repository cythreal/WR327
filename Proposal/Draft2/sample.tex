\documentclass[letterpaper,12pt]{texMemo}
\usepackage[american]{babel}
\usepackage{natbib}
\usepackage{setspace}

\memoto{Roby Conner, Writing 327 Instructor}
\memofrom{Jacques Uber, Oregon State University Student}
\memodate{Feb 12, 2012}
\memosubject{Proposal to investigate imporovements to Tor}

\begin{document}
\begin{center}
\large Memorandum
\end{center}
\singlespace
\maketitle
\section*{Introduction}

It is my goal to investigate how engineers are working to improve Tor. I will focus on the
improvements to latency and congestion, and the methods used to improve user anonymity.

The literature review will include all scholarly articles on Tor found via the Academic Search
Premier database using the keywords Tor, Improvements, Congestion, Fair, Timing Attacks,
Anonymity and published between the years 2009 and 2012.

There will be three section in the document. The first section will introduce core concepts used
to implement Tor. The second will investigate techniques used to compromise user
anonymity. The third will investigate network latency and congestion.

\section*{Background}

    \subsection*{What is Tor?}
    Tor is an overlay network that enables users to use the Internet anonymously. Tor was originally
    developed by the Navy and is used by militaries, journalist, law enforcement, activists, and the
    average internet user\citep[2]{tor:web}. To ensure a users privacy and anonymity Tor uses multiple
    layers of encryption while routing cells of data and is sometimes refered to as the "Onion Router".

    \subsection*{Why is Tor important?}
    Tor has many legititamte uses. ONI, the OpenNet Initiative, reported that "the Middle East and
    North Africa is one of the most heavily censored regions in the world". It also claimed that it
    "conducted tests for technical Internet filtering in all of the countries in the Middle East
    and North Africa between 2008 and 2009. Test results prove that the governments and Internet
    service providers (ISPs) censor content deemed politically sensitive; critical of governments,
    leaders or ruling families; morally offensive; or in violation of public ethics and order." Tor
    can bypass these filters ensuring a freedom of speech and freedom of expression on the Internet.


    \subsection*{Congestion}
    As of 2010 users on the Tor have experienced network delay. The reason why network
    delay was the focus of (\citeauthor[2]{delay}). They ask the question, why is there delay in the network,
    and where is the delay taking place. The routers themselves could be the cause. All traffic that
    goes through the network needs to pass throuh routers. It could be that there are traffic bottle
    necks forming at certain nodes. The methods used to test for what is causing the delays would be
    to set up multiple Onion Routers and measure where bottle necks occur and record how the circuit
    selection algorithm is making node selections.

    Different protocols can cause congestion more than others. This is the focus of .
    There is growing concern that bulk transfer protocols, like Bittorent and other P2P (Peer to
    Peer) protocols, are causing latency sensitive protocols, like ssh and HTTP, to become delayed
    and in some cases hard to use \citep[]{analysis}. This problems are not new to computer networks.
    Major ISPs (Internet Service Providers) have to make sure that their cusomters have the ablity
    to stream music and browse the web while also accomidating other services like FTP (File Transfer
    Protocol) and Bittorent. This coexistance is normally achieved by packet shapers. Packet shapers
    look at traffic (usually the source and destination port of the traffic) and give bandwidth
    priority to latency sensitive protocols. This is not possible on the Tor network. The encryption
    that gives anonymity also stops and QoS (Quality of Service from being applied).

    Reworking how Tor scheduales traffic is a possible solution to Congestion.  A Tor Onion Router
    treats all data equally. A Router will forward data for multiple circuits and it uses a Round Robin
    algorithm to determine which circuit it will service. This means that circuits with data that
    tends to come in bursts will have the same priority as a circuit that has a relativly
    continuous flow of data through it. This is not optimal because data that comes in bursts is usually
    sensative to latency and should take priority over traffic that appears continuous. \citep[]{Tang}
    There have been multiple schedualling schemes proposed to replace the Round Robin schedauler.

\begin{itemize}
    \item
    \citep[]{unfair} This paper explores the mechanisms for handling congestion
    and fairness and proposes a new scheduling algorithm.
    \item
    \citep[]{Tang} This paper proposes a new scheduling algorithm.
    \item
    \citep[]{analysis} Chaabane analysis what the tor network is being used for and where congestion is taking place.
    \item
    \citep[]{delay} Dhungel exposes where and how Tor is unfair to certain kinds of traffic.
    \item
    \citep[]{Bauer} Bauer discusses the consiquences of new routing algorithms and proposes a new
    algorithm.
    \item
    \citep[]{Edman} This paper explores "modifications to Tor's path selection algorithm to help clients avoid
    an AS-level observer".
\end{itemize}


\section*{Research Plan}
Conclusion
    o Summary of Key Points




\bibliographystyle{mla-good}
\bibliography{sample}
\end{document} 
