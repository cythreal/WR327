\documentclass[12pt]{article}
\usepackage[hmargin=1in, vmargin=1in]{geometry}
\usepackage{fancyhdr}
\usepackage{setspace}
\pagestyle{fancy}
\usepackage[small]{caption}
\usepackage{lastpage}
\usepackage{graphicx}
\usepackage{verbatim}
\DeclareGraphicsExtensions{.jpg}
\usepackage{url}

\def\author{Jacques Uber}
\def\title{Problem Statement: Internet Censorship}
\def\date{\today}

\fancyhf{} % clear all header and footer fields
\fancyhead[LO]{\author}
\fancyhead[RO]{\date}
\renewcommand{\headrulewidth}{0pt}
\fancyfoot[C]{\thepage\
                    / 1}

\setcounter{secnumdepth}{0}
\setlength{\parindent}{0pt}
\setlength{\parskip}{4mm}
\linespread{1}
\begin{document}
\underline{
\large{\title}
}

\section{Statement 1}
%To maintain an Internet where network nodes have full interconnectivity with one another, tools must
%be designed that circumvent the filters and barriers built to break connectivity.
The Internet is a network of networks that should have full interconnectivity between all public
nodes.

\section{Statement 2 (But...)}
The combination of faster hardware and market demand has created an Internet unlike the Intranets of
the early and late 1990s. Governments and Corporations can easily block content they deem worthy of
censorship using highly specialized tools developed by western technology firms such as Blue Coat
and Websense.

\section{Statement 3}
If tools that make circumventing filters and firewalls possible are not developed, Governments and
Corporations will find that censoring materials is easy. Internet users will risk being punished for
the content they post on the internet and certain content will not be viewable due to IP blocks, URL
blocks, and other technical manifestations of censorship; the greater Internet will not be connected.

\section{Research}
New ways to ensure connectivity on the internet have emerged. Tor, the The Anonymity Network, is a
tool used to bypass national firewalls and reach content that has been censored. It is also used to post
anonymously to blogs and social media sites. As Governments have become more aware of this technology, the
effectiveness of Tor has broken down. In January 2011 Iran blocked all Tor traffic and stopped
anyone from using Tor from within the networks of the major Iranian ISPs (Internet Service
Providers). China systematically blocks Tor Relays stopping their population from using the network
to connect to servers that are on their black list. Tor has also been suffering from latency and
bandwidth issues due to use of the BitTorrent protocol within it's relays. I will investigate the
ways in which the Tor protocol is becoming more resilient to being identified and blocked by
Governments and ways in which it is improving its QoS (Quality of Service).
\end{document}
